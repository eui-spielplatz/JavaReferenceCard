\documentclass[11pt, a4paper, landscape]{article}
\usepackage[ngerman]{babel}
\usepackage[utf8]{inputenc}
\usepackage[T1]{fontenc}
\usepackage{fancyhdr}
\usepackage{listings}
\usepackage{a4wide}
\usepackage[dvipsnames]{xcolor}
\usepackage{color}
\usepackage{multicol}
\usepackage{color}
\usepackage{xcolor}
\usepackage{underscore}
\usepackage{adjustbox}
\usepackage{lmodern}
\usepackage[margin=1.8cm, lmargin=0.8cm, rmargin=0.8cm, bmargin=1.5cm]{geometry}
\usepackage{struktex}

\definecolor{pageyellow}{HTML}{FFFFF0}
\definecolor{pagegreen}{HTML}{F0FFF0}
\definecolor{pageblue}{HTML}{F0F0FF}
\definecolor{pagegray}{HTML}{F0F0F0}
\definecolor{pagered}{HTML}{FFF0F0}
\definecolor{pageorange}{HTML}{FFF8F0}

\definecolor{borderyellow}{HTML}{C0C080}
\definecolor{bordergreen}{HTML}{80C080}
\definecolor{borderblue}{HTML}{8080C0}
\definecolor{bordergray}{HTML}{808080}
\definecolor{borderred}{HTML}{C08080}
\definecolor{borderorange}{HTML}{C0A080}

\setlength{\fboxrule}{0.5mm}

\pagestyle{fancy}
\lhead{Java Reference Card}
\chead{SS 2016}
\rhead{Informatik I, Teil 1 + 2}
\pagenumbering{gobble}

\lstset{
	basicstyle=\footnotesize,
	breaklines=true,
	captionpos=t,
	extendedchars=true,
	frame=none,
	keepspaces=true,
	keywordstyle=\color{BrickRed},
	commentstyle=\color{Blue},
	identifierstyle=\color{Black},
	language=Java,
	numbers=none,
	numbersep=5pt,
	numberstyle=\color{black},
	rulecolor=\color{black},
	showstringspaces=false,
	showspaces=false,
	showtabs=false,
	stepnumber=5,
	stringstyle=\color{Thistle},
	tabsize=2,
	firstnumber=1,
	texcl=true,
	basicstyle=\scriptsize\ttfamily
}

\newcommand{\fancyheader}[1]{
	\centerline{\sffamily \textbf{ \large #1}}
}

\newenvironment{fancybox}[2]
{
	\begin{adjustbox}{valign=m,center=0.31\textwidth,margin=1ex,minipage=[t][][t]{0.31\textwidth},bgcolor=#1,cfbox=#2 1.5pt 0pt 1pt}
	\begin{centering}
}
{
	\end{centering}
	\end{adjustbox}
}

\begin{document}

%%%%%%%%%%%%%%%%%%%%%%%%%%%%%%%%%%%%%%%

\begin{multicols}{3}
\raggedcolumns

\begin{fancybox}{pagered}{borderred}
\fancyheader{Object}
\lstinputlisting{Object.java}
\end{fancybox}

\begin{fancybox}{pageyellow}{borderyellow}
\fancyheader{Iterable}
\lstinputlisting{Iterable.java}
\end{fancybox}

\begin{fancybox}{pageyellow}{borderyellow}
\fancyheader{Iterator}
\lstinputlisting{Iterator.java}
\end{fancybox}

\begin{fancybox}{pageyellow}{borderyellow}
\fancyheader{Comparable}
\raggedright{
Definiert Totalordnung auf implementierende Objekte.
}
\lstinputlisting{Comparable.java}
\end{fancybox}

\begin{fancybox}{pageyellow}{borderyellow}
\fancyheader{Comparator}
\lstinputlisting{Comparator.java}
\end{fancybox}

\begin{fancybox}{pageyellow}{borderyellow}
\fancyheader{Collection}
\lstinputlisting{Collection.java}
\end{fancybox}

\begin{fancybox}{pageblue}{borderblue}
\fancyheader{Set}
\lstinputlisting{Set.java}

Interface, welches Collection erfüllt; beachte:
\begin{itemize}
\item Ein Set enthält keine zwei Elemente e1 und e2, sodass e1.equals(e2)
\item Ein Set darf sich nicht selbst enthalten
\item Objekte im Set sollten nicht modifiziert werden, sodass nachträglich e1.equals(e2) gilt, sonst undefined behaviour.
\end{itemize}
\end{fancybox}

\begin{fancybox}{pageblue}{borderblue}
\fancyheader{HashSet}
\lstinputlisting{HashSet.java}
\raggedright{
	Verwendet HashMap im Hintergrund. Hat keine Ordnung.
}
\end{fancybox}

\begin{fancybox}{pageblue}{borderblue}
\fancyheader{SortedSet}
\lstinputlisting{SortedSet.java}

\raggedright{
	Die Iteration erfolgt in \emph{aufsteigender} Reihenfolge. Nur interface!
}
\end{fancybox}

\begin{fancybox}{pageblue}{borderblue}
\fancyheader{NavigableSet}
\lstinputlisting{NavigableSet.java}
\end{fancybox}

\begin{fancybox}{pageblue}{borderblue}
\fancyheader{TreeSet}
\lstinputlisting{TreeSet.java}
\end{fancybox}

\begin{fancybox}{pagegray}{bordergray}
\fancyheader{List}
\lstinputlisting{List.java}
\end{fancybox}

\begin{fancybox}{pagegray}{bordergray}
\fancyheader{ArrayList}
\lstinputlisting{ArrayList.java}
\raggedright{
	Im Gegensatz zur LinkedList garantiert ArrayList schnellen Zugriff auf beliebige Indizes.
}
\end{fancybox}

\begin{fancybox}{pagegray}{bordergray}
\fancyheader{Deque}
\lstinputlisting{Deque.java}
\end{fancybox}

\begin{fancybox}{pagegray}{bordergray}
\fancyheader{LinkedList}
\lstinputlisting{LinkedList.java}
\end{fancybox}

\begin{fancybox}{pageorange}{borderorange}
\fancyheader{Map}
\lstinputlisting{Map.java}
\end{fancybox}

\begin{fancybox}{pageorange}{borderorange}
\fancyheader{Map.Entry}
\lstinputlisting{MapEntry.java}
\end{fancybox}

\begin{fancybox}{pageorange}{borderorange}
\fancyheader{HashMap}
\lstinputlisting{HashMap.java}
\end{fancybox}

\begin{fancybox}{pageorange}{borderorange}
\fancyheader{SortedMap}
\lstinputlisting{SortedMap.java}
\end{fancybox}

\begin{fancybox}{pageorange}{borderorange}
\fancyheader{NavigableMap}
\lstinputlisting{NavigableMap.java}
\end{fancybox}

\begin{fancybox}{pageorange}{borderorange}
\fancyheader{TreeMap}
\lstinputlisting{TreeMap.java}
\end{fancybox}

\begin{fancybox}{pagegreen}{bordergreen}
\fancyheader{String}
\lstinputlisting{String.java}
\end{fancybox}

\begin{fancybox}{pagegreen}{bordergreen}
\fancyheader{Struktogramm}

\begin{struktogramm}(70,60)
	\ifthenelse{2}{1}{Bedingung 1}{ja}{nein}
		\ifthenelse{1}{1}{Bedingung 2}{ja}{nein}
			\assign{Tu nix}
			\change
			\assign{Tu das}
		\ifend
		\change
		\assign{Tu dies}
	\ifend
\end{struktogramm}

\end{fancybox}


\end{multicols}
\end{document}
