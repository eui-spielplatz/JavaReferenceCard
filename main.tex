\documentclass[11pt, a4paper, landscape]{article}
\usepackage[ngerman]{babel}
\usepackage[utf8]{inputenc}
\usepackage[T1]{fontenc}
\usepackage{fancyhdr}
\usepackage{listings}
\usepackage{a4wide}
\usepackage[dvipsnames]{xcolor}
\usepackage{color}
\usepackage{multicol}
\usepackage{color}
\usepackage{xcolor}
\usepackage{underscore}
\usepackage{adjustbox}
\usepackage{lmodern}
\usepackage[margin=1.8cm, lmargin=0.8cm, rmargin=0.8cm, bmargin=1.5cm]{geometry}
\usepackage{struktex}
\usepackage{lastpage}

\definecolor{pageyellow}{HTML}{FFFFF0}
\definecolor{pagegreen}{HTML}{F0FFF0}
\definecolor{pageblue}{HTML}{F0F0FF}
\definecolor{pagegray}{HTML}{F0F0F0}
\definecolor{pagered}{HTML}{FFF0F0}
\definecolor{pageorange}{HTML}{FFF8F0}
\definecolor{pagepurple}{HTML}{FFF0F8}
\definecolor{pagecyan}{HTML}{F0FFFF}

\definecolor{borderyellow}{HTML}{C0C080}
\definecolor{bordergreen}{HTML}{80C080}
\definecolor{borderblue}{HTML}{8080C0}
\definecolor{bordergray}{HTML}{808080}
\definecolor{borderred}{HTML}{C08080}
\definecolor{borderorange}{HTML}{C0A080}
\definecolor{borderpurple}{HTML}{C080A0}
\definecolor{bordercyan}{HTML}{80C0C0}

\setlength{\fboxrule}{0.5mm}

\pagestyle{fancy}
\fancyhf{}
\lhead{Java Reference Card}
\chead{Seite \thepage\ / \pageref{LastPage}}
\rhead{Informatik I, Teil 1 + 2}
\pagenumbering{arabic}

\lstset{
	basicstyle=\footnotesize,
	breaklines=true,
	captionpos=t,
	extendedchars=true,
	frame=none,
	keepspaces=true,
	keywordstyle=\color{BrickRed},
	commentstyle=\color{Blue},
	identifierstyle=\color{Black},
	language=Java,
	numbers=none,
	numbersep=5pt,
	numberstyle=\color{black},
	rulecolor=\color{black},
	showstringspaces=false,
	showspaces=false,
	showtabs=false,
	stepnumber=5,
	stringstyle=\color{Thistle},
	tabsize=2,
	firstnumber=1,
	texcl=true,
	basicstyle=\scriptsize\ttfamily
}

\newcommand{\fancyheader}[1]{
	\centerline{\sffamily \textbf{ \large #1}}
}

\newenvironment{fancybox}[2]
{
	\begin{adjustbox}{valign=m,center=0.31\textwidth,margin=1ex,minipage=[t][][t]{0.31\textwidth},bgcolor=#1,cfbox=#2 1.5pt 0pt 1pt}
	\begin{centering}
}
{
	\end{centering}
	\end{adjustbox}
}

\newenvironment{fancyboxwide}[2]
{
	\begin{adjustbox}{valign=m,center=0.48\textwidth,margin=1ex,minipage=[t][][t]{0.48\textwidth},bgcolor=#1,cfbox=#2 1.5pt 0pt 1pt}
	\begin{centering}
}
{
	\end{centering}
	\end{adjustbox}
}

\begin{document}

%%%%%%%%%%%%%%%%%%%%%%%%%%%%%%%%%%%%%%%

\begin{multicols}{3}
\raggedcolumns

\begin{fancybox}{pageyellow}{borderyellow}
\fancyheader{Iterable}
\lstinputlisting{Iterable.java}
\end{fancybox}

\begin{fancybox}{pageyellow}{borderyellow}
\fancyheader{Collection}
\lstinputlisting{Collection.java}
\end{fancybox}

\begin{fancybox}{pageyellow}{borderyellow}
\fancyheader{Iterator}
\lstinputlisting{Iterator.java}
\end{fancybox}

\begin{fancybox}{pageyellow}{borderyellow}
\fancyheader{Comparable}
\raggedright{
Definiert Totalordnung auf implementierende Objekte.
}
\lstinputlisting{Comparable.java}
\end{fancybox}

\begin{fancybox}{pageyellow}{borderyellow}
\fancyheader{Comparator}
\lstinputlisting{Comparator.java}
\end{fancybox}

\begin{fancybox}{pageblue}{borderblue}
\fancyheader{Set}
\lstinputlisting{Set.java}

Interface, welches Collection erfüllt; beachte:
\begin{itemize}
\item Ein Set enthält keine zwei Elemente e1 und e2, sodass e1.equals(e2)
\item Ein Set darf sich nicht selbst enthalten
\item Objekte im Set sollten nicht modifiziert werden, sodass nachträglich e1.equals(e2) gilt, sonst undefined behaviour.
\end{itemize}
\end{fancybox}

\begin{fancybox}{pageblue}{borderblue}
\fancyheader{HashSet}
\lstinputlisting{HashSet.java}
\raggedright{
	Verwendet HashMap im Hintergrund. Hat keine Ordnung.
}
\end{fancybox}

\begin{fancybox}{pageblue}{borderblue}
\fancyheader{SortedSet}
\lstinputlisting{SortedSet.java}

\raggedright{
	Die Iteration erfolgt in \emph{aufsteigender} Reihenfolge. Nur interface!
}
\end{fancybox}

\begin{fancybox}{pageblue}{borderblue}
\fancyheader{NavigableSet}
\lstinputlisting{NavigableSet.java}
\end{fancybox}

\begin{fancybox}{pageblue}{borderblue}
\fancyheader{TreeSet}
\lstinputlisting{TreeSet.java}
\end{fancybox}

\begin{fancybox}{pagegray}{bordergray}
\fancyheader{List}
\lstinputlisting{List.java}
\end{fancybox}

\begin{fancybox}{pagegray}{bordergray}
\fancyheader{ArrayList}
\lstinputlisting{ArrayList.java}
\raggedright{
	Im Gegensatz zur LinkedList garantiert ArrayList schnellen Zugriff auf beliebige Indizes.
}
\end{fancybox}

\begin{fancybox}{pagegray}{bordergray}
\fancyheader{Deque}
\lstinputlisting{Deque.java}
\end{fancybox}

\begin{fancybox}{pagegray}{bordergray}
\fancyheader{LinkedList}
\lstinputlisting{LinkedList.java}
\end{fancybox}

\begin{fancybox}{pageorange}{borderorange}
\fancyheader{Map}
\lstinputlisting{Map.java}
\end{fancybox}

\begin{fancybox}{pageorange}{borderorange}
\fancyheader{Map.Entry}
\lstinputlisting{MapEntry.java}
\end{fancybox}

\begin{fancybox}{pageorange}{borderorange}
\fancyheader{HashMap}
\lstinputlisting{HashMap.java}
\end{fancybox}

\begin{fancybox}{pageorange}{borderorange}
\fancyheader{SortedMap}
\lstinputlisting{SortedMap.java}
\end{fancybox}

\begin{fancybox}{pageorange}{borderorange}
\fancyheader{NavigableMap}
\lstinputlisting{NavigableMap.java}
\end{fancybox}

\begin{fancybox}{pageorange}{borderorange}
\fancyheader{TreeMap}
\lstinputlisting{TreeMap.java}
\end{fancybox}

\begin{fancybox}{pagegreen}{bordergreen}
\fancyheader{String}
\lstinputlisting{String.java}
\end{fancybox}

\begin{fancybox}{pagered}{borderred}
\fancyheader{Object}
\lstinputlisting{Object.java}
\end{fancybox}

\begin{fancybox}{pagered}{borderred}
\fancyheader{Datentypen}
\raggedright {
Attribute und statische Attribute werden von Java auf vorgegebenen Standardwerte initialisiert. Bei lokalen Variablen geschieht dies nicht!
}

\vspace{3mm}

\centering{
\begin{tabular}{r | c | c | c | l}
	\textbf{Type} & \textbf{Min} & \textbf{Max} & \textbf{Init} & \textbf{Wrapper} \\ \hline
	{\ttfamily byte} & -128 & 127 & 0 & {\ttfamily Byte} \\
	{\ttfamily short} & $-2^{15}$ & $2^{15}-1$ & 0 & {\ttfamily Short} \\
	{\ttfamily int} & $-2^{31}$ & $2^{31}-1$ & 0 & {\ttfamily Integer} \\
	{\ttfamily long} & $-2^{63}$ & $2^{63}-1$ & 0L & {\ttfamily Long} \\
	{\ttfamily float} & $-\infty$ & $\infty$ & 0.0f & {\ttfamily Float} \\
	{\ttfamily double} & $-\infty$ & $\infty$ & 0.0d & {\ttfamily Double} \\
	{\ttfamily boolean} & - & - & false & {\ttfamily Boolean} \\
	{\ttfamily String} & - & - & null & - \\
	{\ttfamily Object} & - & - & null & - \\
\end{tabular}
}

\vspace{3mm}
\raggedright {
	Die Wrapper-Klassen {\ttfamily Byte}, {\ttfamily Short}, {\ttfamily Integer}, {\ttfamily Long} enthalten die statischen Konstanten {\ttfamily MAX_VALUE} und {\ttfamily MIN_VALUE}. Die Konstruktoren aller Wrapperklassen akzeptieren auch einen String! \newline
	{\ttfamily Float} / {\ttfamily Double} enthalten beide die statischen Konstanten {\ttfamily NEGATIVE_INFINITY} und {\ttfamily POSITIVE_INFINITY}.
}
\end{fancybox}

\begin{fancybox}{pagered}{borderred}
\fancyheader{Arrays}
\raggedright {
	Arrays haben ein Konstantes {\ttfamily .length}-Attribut, sind aber keine Klasse. Die Klasse {\ttfamily Arrays} enthält einige statische Array-Hilfsfunktionen:
}
\lstinputlisting{Arrays.java}
\end{fancybox}

\begin{fancybox}{pagered}{borderred}
\fancyheader{Reservierte Schlüsselwörter}
{\ttfamily abstract, assert, boolean, break, byte, case, catch, char, class, const, continue, default, do, double, else, enum, extends, false, final, finally, float, for, goto, if, implements, import, instanceof, int, interface, long, native, new, null, package, private, protected, public, return, short, static, strictfp, super, switch, synchronized, this, throw, throws, transient, true, try, void, volatile, while}
\end{fancybox}

\begin{fancybox}{pagecyan}{bordercyan}
\fancyheader{Math}
\lstinputlisting{Math.java}
\end{fancybox}

\begin{fancybox}{pagegreen}{bordergreen}
\fancyheader{Struktogramm}

\begin{struktogramm}(70,60)
	\ifthenelse{2}{1}{Bedingung 1}{ja}{nein}
		\ifthenelse{1}{1}{Bedingung 2}{ja}{nein}
			\assign{Tu nix}
			\change
			\assign{Tu das}
		\ifend
		\change
		\assign{Tu dies}
	\ifend
\end{struktogramm}
\end{fancybox}

\end{multicols}

\newpage

\begin{fancyboxwide}{white}{bordergray}
\fancyheader{Algorithmus: Bubblesort}
\lstinputlisting{BubbleSort.java}
\end{fancyboxwide}

\begin{multicols}{2}
\begin{fancyboxwide}{white}{bordergray}
\fancyheader{Algorithmus: Binärsuche}
\lstinputlisting{BinarySearch.java}
\end{fancyboxwide}
\end{multicols}

\end{document}
